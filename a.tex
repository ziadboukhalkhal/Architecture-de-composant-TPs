% ============================================
% TP3: Spring Data REST
% ============================================
\section{TP3 : Spring Data REST - Exposition Automatique d'API REST}

\subsection{Introduction}

\begin{tcolorbox}[colback=lightgray,colframe=primaryblue,title={\textbf{Objectif du TP}}]
L'objectif de ce TP est de découvrir et maîtriser Spring Data REST, un module qui permet d'exposer automatiquement des APIs REST à partir de repositories JPA, sans avoir à écrire manuellement les controllers. Cette approche permet de générer rapidement des APIs RESTful complètes suivant les principes HATEOAS.
\end{tcolorbox}

\vspace{0.5cm}

\begin{tcolorbox}[colback=white,colframe=secondaryblue,title={\textbf{Compétences visées}}]
\begin{itemize}[leftmargin=*]
    \item Comprendre le fonctionnement de Spring Data REST et son architecture
    \item Maîtriser les annotations @RepositoryRestResource et @RestResource
    \item Créer des projections personnalisées pour contrôler les données exposées
    \item Gérer les relations entre entités (OneToMany, ManyToOne)
    \item Utiliser HATEOAS pour la navigation entre ressources liées
    \item Personnaliser les endpoints et les chemins d'accès
    \item Créer des requêtes de recherche personnalisées (custom queries)
    \item Configurer une base de données H2 en mémoire
\end{itemize}
\end{tcolorbox}

\subsection{Déroulement du TP}

\subsubsection{Description}

Ce TP consistait à créer une application de gestion d'articles et de catégories en utilisant Spring Data REST. Les principales étapes réalisées incluent :

\begin{enumerate}[leftmargin=*]
    \item \textbf{Création des entités JPA} :
    \begin{itemize}
        \item Entité Article avec ses propriétés (description, price, quantity)
        \item Entité Categorie avec relation OneToMany vers Article
        \item Relation bidirectionnelle entre Article et Categorie
    \end{itemize}
    
    \item \textbf{Configuration des repositories} :
    \begin{itemize}
        \item ArticleRepository avec @RepositoryRestResource personnalisé
        \item CategorieRepository pour la gestion des catégories
        \item Requête personnalisée findByCategorie\_Categorie avec @RestResource
    \end{itemize}
    
    \item \textbf{Création de projections} :
    \begin{itemize}
        \item ArticleDTO en tant que projection pour personnaliser l'affichage
        \item Utilisation de @Value pour renommer les champs exposés
        \item Inclusion des informations de la catégorie dans la projection
    \end{itemize}
    
    \item \textbf{Configuration de la base H2} :
    \begin{itemize}
        \item Base de données en mémoire pour le développement
        \item Console H2 activée pour la visualisation des données
        \item Initialisation automatique avec CommandLineRunner
    \end{itemize}
    
    \item \textbf{Test des endpoints générés} :
    \begin{itemize}
        \item Endpoints CRUD automatiques pour Article et Categorie
        \item Endpoint de recherche personnalisé byCategorie
        \item Navigation HATEOAS entre les ressources
    \end{itemize}
\end{enumerate}

L'avantage principal de Spring Data REST est qu'il génère automatiquement tous les endpoints REST (GET, POST, PUT, PATCH, DELETE) sans écrire de code dans les controllers.

\subsubsection{Prises d'écran}

\begin{figure}[H]
    \centering
    \fbox{\includegraphics[width=0.85\textwidth]{tp3_structure.png}}
    \caption{Structure du projet Spring Data REST}
\end{figure}

\begin{tcolorbox}[colback=lightgray,colframe=successgreen,title={\textbf{Explication}}]
\textbf{Ce qui a été fait :} Création de la structure du projet avec les entités JPA, les repositories et la projection.

\textbf{Résultat obtenu :} ✓ Architecture simplifiée sans couche controller grâce à Spring Data REST qui génère automatiquement les endpoints.
\end{tcolorbox}

\vspace{1cm}

\begin{figure}[H]
    \centering
    \fbox{\includegraphics[width=0.85\textwidth]{tp3_h2_console.png}}
    \caption{Console H2 - Visualisation des tables}
\end{figure}

\begin{tcolorbox}[colback=lightgray,colframe=successgreen,title={\textbf{Explication}}]
\textbf{Ce qui a été fait :} Accès à la console H2 via \texttt{http://localhost:8080/h2} pour visualiser les tables créées automatiquement par JPA.

\textbf{Résultat obtenu :} ✓ Les tables ARTICLE et CATEGORIE sont correctement créées avec leurs relations. Les données d'initialisation sont présentes.
\end{tcolorbox}

\vspace{1cm}

\begin{figure}[H]
    \centering
    \fbox{\includegraphics[width=0.85\textwidth]{tp3_root_endpoint.png}}
    \caption{Endpoint racine - Liste des ressources disponibles}
\end{figure}

\begin{tcolorbox}[colback=lightgray,colframe=successgreen,title={\textbf{Explication}}]
\textbf{Ce qui a été fait :} Requête GET vers l'endpoint racine \texttt{http://localhost:8080/} pour découvrir toutes les ressources exposées.

\textbf{Résultat obtenu :} ✓ La réponse HATEOAS liste toutes les ressources disponibles :
\begin{itemize}
    \item /ecommerce (articles)
    \item /categories
    \item /profile (métadonnées)
\end{itemize}
\end{tcolorbox}

\vspace{1cm}

\begin{figure}[H]
    \centering
    \fbox{\includegraphics[width=0.85\textwidth]{tp3_articles_list.png}}
    \caption{GET /ecommerce - Liste des articles}
\end{figure}

\begin{tcolorbox}[colback=lightgray,colframe=successgreen,title={\textbf{Explication}}]
\textbf{Ce qui a été fait :} Requête GET vers \texttt{/ecommerce} (chemin personnalisé défini dans @RepositoryRestResource).

\textbf{Résultat attendu :} Liste de tous les articles avec leurs informations et liens HATEOAS.

\textbf{Résultat obtenu :} ✓ Les 5 articles créés lors de l'initialisation sont retournés avec leurs détails et liens de navigation (self, article, categorie).
\end{tcolorbox}

\vspace{1cm}

\begin{figure}[H]
    \centering
    \fbox{\includegraphics[width=0.85\textwidth]{tp3_projection.png}}
    \caption{GET /ecommerce avec projection}
\end{figure}

\begin{tcolorbox}[colback=lightgray,colframe=successgreen,title={\textbf{Explication}}]
\textbf{Ce qui a été fait :} Requête GET vers \texttt{/ecommerce?projection=articleDTO} pour utiliser la projection personnalisée.

\textbf{Résultat obtenu :} ✓ La projection fonctionne correctement :
\begin{itemize}
    \item Les champs sont renommés (desc, quant, cat)
    \item Les informations de la catégorie sont incluses
    \item La réponse est simplifiée selon nos besoins
\end{itemize}
\end{tcolorbox}

\vspace{1cm}

\begin{figure}[H]
    \centering
    \fbox{\includegraphics[width=0.85\textwidth]{tp3_single_article.png}}
    \caption{GET /ecommerce/\{id\} - Article spécifique}
\end{figure}

\begin{tcolorbox}[colback=lightgray,colframe=successgreen,title={\textbf{Explication}}]
\textbf{Ce qui a été fait :} Requête GET vers \texttt{/ecommerce/1} pour récupérer un article spécifique.

\textbf{Résultat obtenu :} ✓ L'article avec ID 1 est retourné avec tous ses détails et les liens HATEOAS vers la catégorie associée.
\end{tcolorbox}

\vspace{1cm}

\begin{figure}[H]
    \centering
    \fbox{\includegraphics[width=0.85\textwidth]{tp3_search_endpoint.png}}
    \caption{Endpoint de recherche personnalisé}
\end{figure}

\begin{tcolorbox}[colback=lightgray,colframe=successgreen,title={\textbf{Explication}}]
\textbf{Ce qui a été fait :} Accès à \texttt{/ecommerce/search} pour découvrir les méthodes de recherche disponibles.

\textbf{Résultat obtenu :} ✓ L'endpoint \texttt{byCategorie} est exposé et accessible. Le lien templated indique comment l'utiliser.
\end{tcolorbox}

\vspace{1cm}

\begin{figure}[H]
    \centering
    \fbox{\includegraphics[width=0.85\textwidth]{tp3_search_by_category.png}}
    \caption{Recherche d'articles par catégorie}
\end{figure}

\begin{tcolorbox}[colback=lightgray,colframe=successgreen,title={\textbf{Explication}}]
\textbf{Ce qui a été fait :} Requête GET vers \texttt{/ecommerce/search/byCategorie?categorie=CATEGORIE\_1} pour trouver tous les articles d'une catégorie.

\textbf{Résultat obtenu :} ✓ La recherche personnalisée fonctionne. Les articles Article\_1 et Article\_2 appartenant à CATEGORIE\_1 sont retournés.
\end{tcolorbox}

\vspace{1cm}

\begin{figure}[H]
    \centering
    \fbox{\includegraphics[width=0.85\textwidth]{tp3_categories_list.png}}
    \caption{GET /categories - Liste des catégories}
\end{figure}

\begin{tcolorbox}[colback=lightgray,colframe=successgreen,title={\textbf{Explication}}]
\textbf{Ce qui a été fait :} Requête GET vers \texttt{/categories} pour récupérer toutes les catégories.

\textbf{Résultat obtenu :} ✓ Les 3 catégories créées sont listées avec leurs liens vers les articles associés.
\end{tcolorbox}

\vspace{1cm}

\begin{figure}[H]
    \centering
    \fbox{\includegraphics[width=0.85\textwidth]{tp3_category_articles.png}}
    \caption{Navigation HATEOAS - Articles d'une catégorie}
\end{figure}

\begin{tcolorbox}[colback=lightgray,colframe=successgreen,title={\textbf{Explication}}]
\textbf{Ce qui a été fait :} Suivi du lien HATEOAS \texttt{/categories/1/articles} pour récupérer tous les articles d'une catégorie spécifique.

\textbf{Résultat obtenu :} ✓ La navigation HATEOAS fonctionne parfaitement. Les articles de CATEGORIE\_1 sont accessibles via le lien de relation.
\end{tcolorbox}

\vspace{1cm}

\begin{figure}[H]
    \centering
    \fbox{\includegraphics[width=0.85\textwidth]{tp3_post_article.png}}
    \caption{POST /ecommerce - Création d'un article}
\end{figure}

\begin{tcolorbox}[colback=lightgray,colframe=successgreen,title={\textbf{Explication}}]
\textbf{Ce qui a été fait :} Requête POST vers \texttt{/ecommerce} avec un corps JSON pour créer un nouvel article.

\textbf{Résultat obtenu :} ✓ Article créé automatiquement avec HTTP 201 CREATED. Spring Data REST génère l'ID et retourne l'article créé avec ses liens.
\end{tcolorbox}

\vspace{1cm}

\begin{figure}[H]
    \centering
    \fbox{\includegraphics[width=0.85\textwidth]{tp3_patch_article.png}}
    \caption{PATCH /ecommerce/\{id\} - Mise à jour partielle}
\end{figure}

\begin{tcolorbox}[colback=lightgray,colframe=successgreen,title={\textbf{Explication}}]
\textbf{Ce qui a été fait :} Requête PATCH vers \texttt{/ecommerce/2} pour modifier uniquement le prix de l'article.

\textbf{Résultat obtenu :} ✓ La mise à jour partielle fonctionne. Seul le champ "price" est modifié, les autres champs restent inchangés.
\end{tcolorbox}

\vspace{1cm}

\begin{figure}[H]
    \centering
    \fbox{\includegraphics[width=0.85\textwidth]{tp3_delete_article.png}}
    \caption{DELETE /ecommerce/\{id\} - Suppression d'un article}
\end{figure}

\begin{tcolorbox}[colback=lightgray,colframe=successgreen,title={\textbf{Explication}}]
\textbf{Ce qui a été fait :} Requête DELETE vers \texttt{/ecommerce/5} pour supprimer l'article avec l'ID 5.

\textbf{Résultat obtenu :} ✓ L'article est supprimé avec succès (HTTP 204 NO CONTENT). Vérification effectuée ensuite avec GET.
\end{tcolorbox}

\subsubsection{Résultats \& observations}

\begin{tcolorbox}[colback=white,colframe=successgreen,title={\textbf{Résultats obtenus}}]
\begin{itemize}[leftmargin=*]
    \item[\textcolor{successgreen}{✓}] Génération automatique de tous les endpoints REST (GET, POST, PUT, PATCH, DELETE)
    \item[\textcolor{successgreen}{✓}] Relations JPA correctement exposées avec navigation HATEOAS
    \item[\textcolor{successgreen}{✓}] Projections personnalisées fonctionnelles pour contrôler les données exposées
    \item[\textcolor{successgreen}{✓}] Requêtes de recherche personnalisées opérationnelles
    \item[\textcolor{successgreen}{✓}] Base de données H2 configurée et fonctionnelle
    \item[\textcolor{successgreen}{✓}] Console H2 accessible pour le débogage
    \item[\textcolor{successgreen}{✓}] Initialisation automatique des données via CommandLineRunner
    \item[\textcolor{successgreen}{✓}] Pagination et tri disponibles automatiquement sur les collections
\end{itemize}
\end{tcolorbox}

\vspace{0.5cm}

\begin{tcolorbox}[colback=white,colframe=warningorange,title={\textbf{Difficultés rencontrées}}]
\begin{enumerate}[leftmargin=*]
    \item \textbf{Problème :} Relations bidirectionnelles causant des erreurs de sérialisation JSON (boucle infinie)
    
    \textbf{Cause :} Relation OneToMany et ManyToOne entre Article et Categorie créant une référence circulaire
    
    \textbf{Solution adoptée :} Utilisation de projections pour contrôler exactement quels champs sont exposés, évitant ainsi les boucles infinies
    
    \item \textbf{Problème :} La projection n'était pas appliquée par défaut
    
    \textbf{Cause :} Le paramètre excerptProjection n'était pas configuré dans @RepositoryRestResource
    
    \textbf{Solution adoptée :} Ajout de \texttt{excerptProjection = ArticleDTO.class} dans l'annotation @RepositoryRestResource
    
    \item \textbf{Problème :} La console H2 n'était pas accessible
    
    \textbf{Cause :} Configuration manquante dans application.properties
    
    \textbf{Solution adoptée :} Ajout de \texttt{spring.h2.console.enabled=true} et \texttt{spring.h2.console.path=/h2}
    
    \item \textbf{Problème :} Les données d'initialisation n'étaient pas persistées
    
    \textbf{Cause :} Les entités n'étaient pas sauvegardées dans le bon ordre (relation catégorie → article)
    
    \textbf{Solution adoptée :} Modification du CommandLineRunner pour sauvegarder d'abord les catégories, puis associer les articles
\end{enumerate}
\end{tcolorbox}

\vspace{0.5cm}

\begin{tcolorbox}[colback=lightgray,colframe=primaryblue,title={\textbf{Solutions techniques adoptées}}]
\begin{itemize}[leftmargin=*]
    \item \textbf{@RepositoryRestResource} : Personnalisation des endpoints avec \texttt{collectionResourceRel} et \texttt{path} pour des URLs plus intuitives
    
    \item \textbf{Projections avec @Projection} : Contrôle fin des données exposées en créant des vues personnalisées des entités
    
    \item \textbf{@RestResource} : Personnalisation des méthodes de recherche avec des noms et chemins explicites
    
    \item \textbf{HATEOAS} : Navigation facilitée entre ressources grâce aux liens \_links générés automatiquement
    
    \item \textbf{CommandLineRunner} : Initialisation élégante des données de test au démarrage de l'application
    
    \item \textbf{H2 Console} : Outil de débogage précieux pour visualiser les tables et vérifier les données
    
    \item \textbf{@Value dans les projections} : Renommage et transformation des champs exposés pour une API plus claire
\end{itemize}
\end{tcolorbox}

\subsubsection{Lien GitHub}

\begin{tcolorbox}[colback=lightgray,colframe=primaryblue]
\centering
Le code source complet de ce TP est disponible sur GitHub :

\vspace{0.3cm}

\href{https://github.com/[votre-username]/tp3-spring-data-rest}{\textbf{\large https://github.com/[votre-username]/tp3-spring-data-rest}}
\end{tcolorbox}

\newpage